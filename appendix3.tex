\chapter{アプリケーション層に関する補講}

\section{魔法使いの弟子シンドローム}

バイナリ思考のプロトコルであるTFTPは、トランスポート層にUDPを使用する。そのため転送したデータのチャンクはアプリケーション側で到着順を管理する必要がある。だが、初期の実装はその処理に問題があり、それに起因する、魔法使いの弟子シンドロームというバグがあった。本省では、それについて説明をおこなう。

\subsection{Der Zauberlehrling}

ゲーテのバラードに、「魔法使いの弟子(Der Zauberlehrling)」という作品がある。デュカスの「交響詩 魔法使いの弟子」の原典となり、ディズニーアニメのファンタジアで、ミッキーマウスがこの「魔法使いの弟子」となった。\footnote{本書の出版時点では実写映画化もされている}

そのあらすじは、以下の通りである。

\begin{quotation}
魔法使いは、弟子に水汲みをするよう言い残して出ていった。
\\
弟子は、面倒なので箒に魔法をかけ、水汲みをさせる。そのうちに水汲みは終わったが、弟子は箒を止める魔法を知らなかった。箒はそのまま水汲みを続け、あふれた水で攻防は水浸しとなった。

弟子は最後の手段として、鉈で箒を壊し、水汲みを止めようとした。だが、両断されたそれぞれが水汲みをつづけるので、先ほどの倍の早さで水浸しになっていく。
\\
手のつけようがなくなったときに魔法使いが戻ってきて、魔法で箒を止め、水を消し、最後に弟子をしかりつけるのであった。
\end{quotation}


TFTPには、RFC1123で指摘され、この詩から名前を採った「魔法使いの弟子シンドローム」というアルゴリズムバグが存在した。

TFTPの説明では、重複ACKを受信したら、その重複ACKに対応したブロックを再送してはならないと書いた。古い実装では、この点が考慮されていなかった。

では、重複ACKに対してブロックを再送したらどうなるか、それを説明しよう。

\subsection{魔法使いの弟子問題のプロセス}

それは、重複ACKに対応するブロックを再送したら、それ以降の全てのパケットが重複して送信されるためである。重複ACKに対して対応するブロック番号のDATAセグメントを送信し直すことは、鉈で箒をまっぷたつにすることに相当するわけだ。重複ACKがなぜ生じるか。それは、ACKが遅延したことで送信側がDATAパケットを再送するからである。それによって、遅延したACKと再送したDATAパケットに対してのACKの二つが到着する。そのそれぞれに対して対応するブロックを送信することで、それ以降は、同じブロック番号のDATAパケットを二重に送信することになってしまう。

ACKが遅延した場合のシナリオを、RFC1123より引用してみよう。

\begin{table}[hbtp] \caption{魔法使いの弟子シンドローム} \label{rfc1123}
\begin{center}
{\footnotesize
	\begin{tabularx}{13cm}{lllX} \toprule
	時間 & ホストA(送信側) & ホストB(受信側) & - \\ \midrule
	1 & \shortstack{ ACK X-1受信 \\ DATA X送信} & - & - \\ \hline	
	2 & -  & \shortstack{ DATA X受信 \\ ACK X送信}  & - \\ \hline
	- & - & - & ACK Xが遅延し、ホストAのACK X待ちがタイムアウト\\ \hline
	3 & DATA X再送信 & - & - \\ \hline
	4 & - & \shortstack{DATA Xを再受信 \\ ACK X再送信} & - \\ \hline
	- & - & - & ここで、遅延していた方のACK Xが到着する \\ \hline
	5 & \shortstack{遅延していた ACK X受信 \\ DATA X+1送信} & - & - \\ \hline 	
	6 & - & \shortstack{ DATA X+1受信 \\ ACK X+1送信} & - \\ \hline
	7 & \shortstack{再送に対する ACK X受信 \\ DATA X+1を再送信} & - & - \\ \hline 	
	8 & - & \shortstack{DATA X+1再受信 \\ ACK X+1再送信} & - \\ \hline
	9  & \shortstack{ACK X+1受信 \\ DATA X+2送信} & - & - \\ \hline 	
	10 & - & \shortstack{DATA X+2受信 \\ ACK X+2送信} & - \\ \hline
	11 & \shortstack{ACK X+1再受信 \\ DATA X+2再送信} & - & - \\ \hline 	
	12 & - & \shortstack{DATA X+2再受信 \\ ACK X+2再送信} & - \\ \bottomrule
	\end{tabularx}
}
\end{center}
\end{table}


ACKの遅延は、インターネットプロトコル層以下の輻輳が原因であることがほとんどであり、その状況で魔法使いの弟子シンドロームが生ずると、輻輳を拡大再生産することになる。そのため、現在のTFTPの規格では、重複ACKに対応するブロックを再送信してはならないことになっている。