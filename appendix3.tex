\chapter{アプリケーション層に関する補講}

\section{魔法使いの弟子シンドローム}

バイナリ指向のプロトコルであるTFTPは、トランスポート層にUDPを使用する。そのため転送したデータのチャンクはアプリケーション側で到着順を管理する必要がある。だが、初期の実装はその処理に問題があり、それに起因する、魔法使いの弟子シンドローム(SAS Sorcerer's Apprentice Syndrome)というバグがあった。
本章では、UDPでフローを管理する実装の例として、TFTPの魔法使いの弟子シンドロームを取り上げる。

\subsection{Der Zauberlehrling}

ゲーテのバラードに、「魔法使いの弟子(Der Zauberlehrling)」という作品がある。デュカスの「交響詩 魔法使いの弟子」の原典となり、ディズニーアニメのファンタジアで、ミッキーマウスがこの「魔法使いの弟子」となったし、実写映画化もされている。

そのあらすじは、以下の通りである。

\begin{quotation}
魔法使いは、弟子に水汲みをするよう言い残して出ていった。
\\
弟子は、面倒なので箒に魔法をかけ、水汲みをさせる。そのうちに水汲みは終わったが、弟子は箒を止める魔法を知らなかった。箒はそのまま水汲みを続け、あふれた水で攻防は水浸しとなった。

弟子は最後の手段として、鉈で箒を壊し、水汲みを止めようとした。だが、両断されたそれぞれが水汲みをつづけるので、先ほどの倍の早さで水浸しになっていく。
\\
手のつけようがなくなったときに魔法使いが戻ってきて、魔法で箒を止め、水を消し、最後に弟子をしかりつけるのであった。
\end{quotation}

魔法使いの弟子シンドロームは、RFC1123の4.2.3 SPECIFIC ISSUESでで指摘されている。
\footnote{https://tools.ietf.org/html/rfc1123}
これは、重複ACKの処理に関するアルゴリズムバグである。
TFTPは、アプリケーション層のプロトコルとして、ACKを遣り取りする。TCPのACKでは無いことに注意してほしい。

ACKを送るとき、重複ACKを受信したら、その重複ACKに対応したブロックを再送してはならない。だが、古い実装では、この点が考慮されていなかった。

\subsection{魔法使いの弟子シンドロームのプロセス}

TFTPは、重複ACKに対応するブロックを再送したら、それ以降の全てのパケットが重複して送信される。
重複ACKに対して対応するブロック番号のDATAセグメントを送信し直すことは、鉈で箒をまっぷたつにすることに相当するわけだ。
重複ACKがなぜ生じるかというと、ACKが遅延したことで送信側がDATAパケットを再送するからである。
それによって、遅延したACKと再送したDATAパケットに対してのACKの二つが到着する。
そのそれぞれに対して、TFTPサーバが対応するブロックを送信することで、同じブロック番号のDATAパケットを二重に送信することになってしまう。

ACKが遅延した場合のシナリオを、RFC1123より引用してみよう。

\begin{table}[hbtp] \caption{魔法使いの弟子シンドローム} \label{rfc1123}
\begin{center}
{\footnotesize
	\begin{tabularx}{13cm}{lllX} \toprule
	時間 & ホストA(送信側) & ホストB(受信側) & - \\ \midrule
	1 & \shortstack{ ACK X-1受信 \\ DATA X送信} & - & - \\ \hline	
	2 & -  & \shortstack{ DATA X受信 \\ ACK X送信}  & - \\ \hline
	- & - & - & ACK Xが遅延し、ホストAのACK X待ちがタイムアウト\\ \hline
	3 & DATA X再送信 & - & - \\ \hline
	4 & - & \shortstack{DATA Xを再受信 \\ ACK X再送信} & - \\ \hline
	- & - & - & ここで、遅延していた方のACK Xが到着する \\ \hline
	5 & \shortstack{遅延していた ACK X受信 \\ DATA X+1送信} & - & - \\ \hline 	
	6 & - & \shortstack{ DATA X+1受信 \\ ACK X+1送信} & - \\ \hline
	7 & \shortstack{再送に対する ACK X受信 \\ DATA X+1を再送信} & - & - \\ \hline 	
	8 & - & \shortstack{DATA X+1再受信 \\ ACK X+1再送信} & - \\ \hline
	9  & \shortstack{ACK X+1受信 \\ DATA X+2送信} & - & - \\ \hline 	
	10 & - & \shortstack{DATA X+2受信 \\ ACK X+2送信} & - \\ \hline
	11 & \shortstack{ACK X+1再受信 \\ DATA X+2再送信} & - & - \\ \hline 	
	12 & - & \shortstack{DATA X+2再受信 \\ ACK X+2再送信} & - \\ \bottomrule
	\end{tabularx}
}
\end{center}
\end{table}


ACKの遅延は、インターネットプロトコル層以下の輻輳が原因であることが多い。
その状況で魔法使いの弟子シンドロームが生ずると、インターネットプロトコル層以下の輻輳を拡大再生産することになる。
そのため、現在のTFTPの規格では、重複ACKに対応するブロックを再送信してはならない。