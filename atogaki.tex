\chapter{あとがき}

\section*{メイ・カートミル(仮名)とマージ・ニコルス(仮名)の秋葉原での会話}

\begin{quotation}
\noindent
{\bf メイ}「ゲーム用のNICってオカルトですよねー」 \\
{\bf マージ}「素直にIntelあたりののNICを{\bf つんでれ}ばいいのに」 \\
{\bf メイ}「IntelのNICって{\bf ツンデレ}だったんですね」 \\
{\bf マージ}「べっ別にあんたのために物理層制御してやってるわけじゃないんだからねっ」 \\
\end{quotation}

\section*{初版後書き コミケ前日の朝}

現在、2010年8月12日の午前8時30分です。同人誌の後書きでよく、前日や当日朝のタイムスタンプになっているものを見ましたが……自分で作ってみて判りました。コピー本は本当にギリギリまで修正を繰り返してしまうものです。さっきまで、画像の回り込み行数の微調整やら校正やらやってました。

まとめてみて、予想よりページ数が多くなり、いささか頭を抱えています。中綴じ製本コピー本としてはほぼ限界レベルです。うちにあるホチキスで中綴じ100ページは作ったことがあるのでおそらく大丈夫とは思いますが、PDF販売も考えましたよ。

「ささめきこと」で袋とじ製本をするシーンがありましたが、あれは…もうやりたくないですね。

拙い内容かつ、いまどきのネットワークプログラミングでは押さえておくべきマルチキャストについてはまったく触れていこともいささか心残りです。とはいえ、今はただ、このような本を手に取ってくれた方への感謝をするばかりです。

TCP/IP理解の一助となりますように。

\subsection*{追伸}
イーサネットでループを作ってはいけない理由が、心から理解できている人が増えるともっと嬉しいです。

\section*{改訂版後書き}
前著からかなりの書き直し修正を加えて、今回改訂版として出し直すことにしました。前の版を見ながら、自分で頭を抱え、コミケ4日目(笑)から書き直しをはじめてようやく後書きまでたどり着きました。行間が狭くて読みづらい、図や表が少ないという意見をいただいていたので、行間設定は0.5から0.8へ、図版もそれなりに追加しています。そのせいもあって140ページの厚さになってしまいました。それでも、かなりの省略がありますので、参考文献に挙げた本に手を出すためのとっかかりくらいに考えていただいた方がよいかと思います(気弱)

IPv6のテキストを、と言うリクエストもいただきましたが、そちらは自分で使って勉強中のレベルです(苦笑)。むしろ、そういう同人誌が出てたら買いに行きます…あ、昔買った気がする。ただ、IPv6のテキストは、IPv4によるTCP/IPの理解を前提とした書き方がされていることが多いので、その本を読む準備段階と言うことで。

本書はネットワークアプリケーションをプログラムするプログラマ向けのテキスト、というコンセプトです。ですが、ページの都合(印刷費というリアルな問題が…)の都合で、現在のネットワークにおいて意識せざるを得ない、ファイアウオールやNAT、SSL/TLS、プロキシといった、古典的な教科書では触れられていないジャンルを割愛せざるを得ませんでした。

IPv6導入の代わりに、キャリアグレードNATで済まそうという、ネットワーク屋、サーバ屋からみるとあまりよろしくない動きがある中、エンドツーエンドを考えるのに必要な事柄については、改めてテキストとしてまとめていきたいです。

\section*{第三版後書き}
改訂版を出してから1年を経て、また改訂して第三版として出すことになりました。幸いにも、同人誌なのに正面からTCP/IPの基礎を解説するというコンセプトを、好意的に受け取ってくださる方が多く、コミックマーケット80では見本誌として残った一冊をこれも多くの方に見ていただきました。

そうやって見ていただくきっかけになったであろう表紙の女の子についての話を少しばかり。まずは、目を引く表紙を描いてくれたみるふぃー氏に感謝です。改訂版では謝辞を忘れて申し訳ない次第。

本書に始まって、拙著の同人版TCP/IPシリーズのカバーガールとして活躍してもらっている彼女ですが、髪をまとめるUTPケーブルに木の洗濯ばさみがつけてあることにお気づきでしょうか。もし彼女の今後を気にしていただけるなら、シリーズでご購入いただけると更に嬉しいです、と、軽く宣伝もさせていただきましょう。

改訂版で「そろそろIPv4完売」と書いたら、「こうどなじょうほうせん」をする余裕もなく、IPv4は完売しました。ですが、まだまだIPv6普及の兆しも、見えないことに、世の中キャリアグレードNATとかそっちのほうにいくのではないかと戦々恐々です。だからIPv4で本を書いていいんだ、とは開き直れませんが、まだまだ大丈夫クマーと、そう思いつつ本書の改訂を終えました。

それでも、いずれかはIPv6と正面から向き合わなければならなくなるでしょう。そのときは、本書もコウモリ本\footnote{オライリーのsendmailは表紙がコウモリなのでコウモリ本と呼ばれる。最初は生cfの書き方を解説していたが、次の版から「生cf生成ツール」の解説本となった}レベルの改訂を行いたいものです。

TCP/IPを「どうしてこうなった」という部分から説く、誰得な一冊にできたのではないか、そんな自負を持ちつつ第三版の〆とさせていただきます。



\section*{第四版後書き}
第三版から5年、ようやく、IPv6についての情報を盛り込んだ、あらたな改訂版を出すことが出来ました。ひとまず、後書きまでたどり着けて安堵しています。

ただ、コウモリ本ほどの改訂になったのかはわかりませんが、大きな改訂となったことは間違いがありません。

TCP/IPを最初から学ぼうとしたとき、ひとつの大きな問題があります。それは、IPv6の扱いです。TCP/IPの入門書では、IPv4とIPv6の関係や違いなど、過去から現在のインターネットプロトコルを織り込んだTCP/IPについて学べるテキストは、寡聞にして見る機会を得ていません。一方、IPv6について解説する書籍は、基本的にTCP/IP、それもIPv4の知識があることを前提としています。

それらに対して、自分なりの答えを出したのがこの一冊です。

最後になりますが、表紙を担当していだだきましたH-甘さんに感謝を捧げます。この本に潤いと彩りをありがとう。

\begin{flushright}
2016年8月14日 \\
ありす ゆう
\end{flushright}

まずは表紙のH-甘さん、かわいい表紙をありがとうございます。

気がつくと、この本にも深く関わっていました。TCP/IPの入門書といういささか無茶な同人誌ですが、それによって商業売れ筋の本とはまた違うアプローチができたんじゃないかなって考えています。

これから、IPv6を使わなければならない世界がやってきます。それは、インターネットの使い方そのものを変革する可能性があります。その流れに対してお兄ちゃんはエンライテンドになるのかレジスタンスになうのか。陣取り合戦はもう始まってるんじゃないでしょうか。



\begin{flushright}
2016年8月14日 \\
インフラエンジニアの毒な妹 \\
\end{flushright}


%\newpage
% ここまでで160ページ鳴ったのでブランクなし
% 1ページブランクを入れる

\thispagestyle{empty}
\mbox{}
\newpage
\clearpage


\thispagestyle{empty}

\vspace*{\fill}
\begin{tabular}{ll} \toprule
筆者 & ありす ゆう インフラエンジニアの毒舌な妹 \\
発行 & AliceSystem \\
連絡先 & aliceyou@alicesystem.net \\
URL & http://aliceyou.air-nifty.com/onesan/ \\
初版発行日 & 2010年8月14日 \\
改訂版発行日 & 2010年12月31日 \\
第三版発行日 & 2011年12月31日 \\
第三版発行日 & 2016年8月14日 \\
印刷所 & (株)ポプルス \\ \bottomrule
\end{tabular}