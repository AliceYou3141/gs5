\begin{thebibliography}{99}
\item
	Stevens,W.Richard
	橘康雄 訳 井上尚司 監訳
	2000
	『詳解TCP/IP vol1 プロトコル』
	東京:ピアソン・エデュケーション
\item
    Hunt,Craig.
	林 秀幸 訳、村井 純、土本 康生 監訳
	2003
	『TCP/IPネットワーク管理 第3版』
	東京:オライリー・ジャパン
\item	
	南山智之、佐々木彬夫
	1998
	『TCP/IPとソケット クライアント/サーバ構築法』
	東京:共立出版
\item
	Perlman,Radia
	加藤 朗 監訳
	1995
	『Interconnections ブリッジとルータについて』
	東京:ソフトバンク
\item
	Comer,Douglas
	村井純、楠本博之 訳
	1996
	『第3版 TCP/IPによるネットワーク構築 Vol.I 原理・プロトコル・アーキテクチャ』
	東京:共立出版
\item
	Comer,Douglas Stevens,David
	村井純、楠本博之 訳
	1995
	『TCP/IPによるネットワーク構築 Vol.II 設計・実装・内部構造』
	東京:共立出版
\item
	Davidson,John M.
	後藤滋樹、村上健一郎、野島久雄 訳
	1991
	『はやわかりTCP/IP』
	東京:共立出版
\item
	笠野英松 監修
	マルチメディア通信研究会 編
	1998
	『ポイント図解式 インターネットRFC事典』
	東京:アスキー
\item
	奥山徹
	2004
	『TCPのしくみと実装』
	東京:CQ出版社
\item
	網野衛二
	2008
	『3分間ルーティング基礎講座』
	東京:技術評論社
\item
	小俣光之、種田元樹
	2011
	『Linuxネットワークプログラミングバイブル』
	東京:共和システム
\item
	城戸正博
	2002
	『ジョーク無しでインターネット技術は語れない!【ジョークRFCの本】』
	東京:ラトルズ
\item	
	小野塚謙太
	2009
	『超合金の男 -村上克司伝-』
	東京:アスキー・メディアワークス
\item
	小林 佳和
	1996
	『通信SEハンドブック 徹底解説 LAN工事実戦テクニック LANの基礎知識と統合配線の技術』
	東京:リックテレコム
\item
     Sterling Jr.,Donald J.
	赤木保之 訳
	1998
	『LANケーブリング ベーシックマニュアル TIA/EIA-568Aの実践的解説書』
	東京:リックテレコム
\item
     Spurgeon,Charles E
	柏木由美子 訳
	櫻井豊 監訳
	2000
	『詳説 イーサネット』
	東京:オライリー・ジャパン
\item
    インターフェース編集部 編
    2006
    『Ethernetのしくみとハードウェア設計技法 プロトコル尾詳細からネットワーク対応機器の作成まで』
    東京:CQ出版株式会社	
\item
	Hagan,Silvia
	豊沢 聡 訳
	市原 英也 監訳
	2007
	『IPv6エッセンシャルズ 第2版』
	東京:オライリー・ジャパン
\item
	IRI
	ユビキタス研究所
	2005
	『マスタリングTCP/IP IPv6編』
	東京:オーム社
\item
	遠藤 哲、伊藤 玄蕃、堀口 幹友
	2010
	『激変するTCP/IP』
	ASCII .technologies 2010年10月号
	東京:アスキー・メディアアークス
\item
      Stevens, W.Richard
	篠田陽一訳
	1992
	『UNIXネットワークプログラミング』
	東京:トッパン
\item
     Wood,David.
	大川佳織訳、佐々木雅之、澤野弘幸、千葉猛、鄭隆幸、日比野洋克、平塚伸世、渡部直明 監訳
	2000
	『電子メールプロトコル - 基本・実装・運用』
	東京:オライリー・ジャパン	
\item
      Goethe,Johann Wolfgang von
	L'Apprenti sorcier
	万足卓
	1982年
	『魔法使いの弟子窶燈]釈・ゲーテのバラード名作集』
	東京:三修社
\item
	Waitzman,D.
	SATOH\_Yoshiyuki訳
	1990
	A Standard for the Transmission of IP Datagrams on Avian Carriers Internet Engeneering Task Force RFC1149
	http://www.imasy.or.jp/~yotti/rfc1149ej.txt
\item
     Makakos,L. Lidi,K. Evarts,J. Carrel,D. Simone,D. Wheeler,R
	しいしせねっと訳
	1999
	Method for Transmitting PPP Over Ethernet Internet Engineering Task Force RFC 2516
     http://www.siisise.net/rfc/rfc2516.html
\item
     岡部泰一
	2001
	『イーサネット(その1)イーサネットの規格とアクセス制御方法 3.10BASE5イーサネットの通信モデル』 @IT『ネットワーク技術養成講座 詳説/IPプロトコル』 
	東京:アイティメディア
     http://www.atmarkit.co.jp/fwin2k/network/tcpip006/tcpip03.html
\item
     著者不詳
	2003
	『初めてのギガビット・イーサネット 第3回 技術編:知恵と工夫で確保した伝送距離100メートル』 IT Pro 『一週間で学ネットワークの要点』
	東京:日経BP
     https://itpro.nikkeibp.co.jp/members/NNW/NETPOINT/20031021/3/
\item
    岩崎有平 福井雅章
	2006
	『イーサネットの基本原理(2)、フレーム間ギャップと最短フレーム長の存在意義』 IT PRo 『【Networkゼミナール】イーサネット技術読本』
	東京:日経BP
     http://itpro.nikkeibp.co.jp/article/COLUMN/20060523/238716/
\item
     安藤 雅人
	2004
	『LAN Switch技術 ~冗長化手法とループ防止~』 Internet Week2004発表スライド
	WIDE大学
     http://www.soi.wide.ad.jp/class/20040031/slides/23/index\_39.html
\item
     村井 純
	2003
	『第2回:低レイヤー技術』慶應義塾大学SFC授業資料『インターネット構成法』
	WIDE大学
     http://asari.sol.wide.ad.jp/class/20030021/slides/02/index\_14.html
\item
     株式会社ラインアイ
	年度不詳
	『伝送路符号』編者不詳 『通信基本用語一覧』
	京都:ラインアイ
     http://www.linedye.co.jp/html/term\_denso.html
\item
	Gene 著
	2000
	『イーサネットのフレームフォーマット』
	東京:エンジニアサポート
	http://www.n-study.com/network/frame.htm
\item
	著者不詳
	年度不詳
	Ethernet Frame
	http://www.infocellar.com/networks/ethernet/frame.htm
\item
	著者不詳
	年度不詳
	Novell 802.3 Raw
	http://www.cam.hi-ho.ne.jp/puffin/compendium/J\_EN-FrFNo.html
\item
      Postal,J
	srgia訳
	1981
	INTERNET MESSAGE CONTROL PROTOCOL Internet Engineering Task Force RFC792
     http://srgia.com/docs/rfc792j.html
\item
      Baker,F. ed. 
	k-hig訳
	1995
	Requirements for IP Version 4 Routers Internet Engineering Task Force RFC1812
     http://www2s.biglobe.ne.jp/~hig/rfc/Router01.html
\item
      Mogul,J. Postel,J.
	k-hig訳 
	1985
	Internet Standard Subnetting Procedure Internet Enmgineering Task Force RFC950
	http://www2s.biglobe.ne.jp/~hig/rfc/rfc950j.txt
\item
    著者不詳
	2007
	『ネットワークプログラミング相談室Portr20 http://pc11.2ch.net/test/read.cgi/tech/1186518855/』 
	編者不詳 
	『オラオラ検索 2ちゃんねる検索(キャッシュ)』
    http://oraken.net/2ch/?index=301\&thread\_id=2333\&plain=\&ad\_pos=351
\item
      Postel,J
	k-hig訳
	1980
	User Datagram Protocol Internet Engineering Task Force RFC768
     http://www2s.biglobe.ne.jp/~hig/tcpip/rfc768j.txt
\item
      Rey,Marina del.
	Ishida So訳
	1981
	Transmission Control Protocol DARPA Internet Program Protocol Specification IETF RFC 793
     http://www5d.biglobe.ne.jp/~stssk/rfc/rfc793j.html
\item
     Josnsson,L.E. Fairhurst,G. eds Larzon,L.A. Degermark,M. Pink,S.
	2004
	The Lightweight User Datagram Protocol (UDP-Lite) Internet Engineering Task Force RFC3828
     http://www.ietf.org/rfc/rfc3e828.txt
\item
     デジタルアドバンテージ
	2003
	『2. UDPパケットの構造』、@IT 『基礎から学ぶWindowsネットワーク』
     http://www.atmarkit.co.jp/fwin2k/network/baswinlan013/baswinlan013\_03.hjtml
\item
     津川知朗
	年度不詳 
	『Delayed ACK』
	津川知朗
	『TT Homepage』
     http://www.anarg.jp/~t-tugawa/note/misc/delayed\_ack.html
\item
      Young,Warren
	Mori,Keisuke.訳
	2002
	『第3章:Winsock 中級者向けの議論』 
	『Winsock Programmer's FAQ』
     http://www.kt.rim.or.jp/~ksk/wskfaq-ja/intermediate.html
\item
      網野衛二
	年度不詳
	『第41回 レイヤ4 TCP ウインドウ』
	網野衛二
	『三分間ネットワーキング』
     http://www5e.biglobe.ne.jp/~aji/3min/41.html
\item
      網野衛二
	年度不詳
	『第42回 レイヤ4 TCP 輻輳制御』
	網野衛二
	『三分間ネットワーキング』
     http://www5e.biglobe.ne.jp/~aji/3min/41.html

\item
      Braden,R ed
	k-hig訳
	1989
	Requiremt for Internet Hosts 窶骭€ Application and Support, Internet Engineering Task Force RFC1123
      http://www2s.biglobe.ne.jp/~hig/tcpip/HostRes\_Appl.html
\item
      Sollins,K
	岡橋一輝訳
	1992
	The TFTP Protocol (Revision 2) Internet Engineering Task Force RFC 1350
     http://www5d.biglobe.ne.jp/~stssk/nro/rfc1350j.txt
\item
	Honoria
	『参考文献の書き方』
	奇食ハッチポッチ
	http://ymnnmy.nobody.jp/quotation.html
\item
	IRI・ユビキタス研究所 共著
	2006
	『マスタリングTCP/IP IPv6編』
	東京:オーム社開発局
\item
	2001
	『BSD magazien N0.8』
	東京:株式会社アスキー
\item
	日系NETOWRK 編
	2012
	『絶対わかる!ネットワーク設計調入門増補改訂版』
	東京:日系BP社
\item
	Miller,Mark A. 著
	トップスタジオ 訳
	宇夫 陽次朗 監修
	1999
	『IPv6入門』
	東京:翔泳社
\item
	ネットテクノロジーラボ 著
	1999
	『最新技術解説 入門IPv6』
	東京:技術評論社
\item
	Huitema,Christian 著
	松島栄樹 WIDEプロジェクトIPv6分科会 訳
	村井純 監修
	1996
	『IPv6次世代インターネットプロトコル 』
	東京:プレンティスホール出版
\item
	Leffer,S.J Mckusick,M.K. Karels,M.J. Quarerman,J.S. 著
	中村明 相田仁 計宇生 小池汎平 共訳
	1991
	『UNIX 4.3BSDの設計と実装 』
	東京:丸善株式会社
\item
	吉澤英明 神田充 高宮紀明 関谷勇司 江崎浩 村井純
	2002
	『USAGIプロジェクトによるIPv6基本ソフトウエアの開発』
	WIDEプロジェクト
	http://hiroshi1.h
\item
	D.Borman Berkeley software Design S.Deering Cisco  R.Hinden Nokia
	1999
	 IP Jumbograms RC2675
	 https://tools.ietf.org/html/rfc2675
\item
	J.Postel ISI
	1981
	INTERNET CONTROL MESSAGE PROTOCOL
	https://tools.ietf.org/html/rfc792
\item
	A.Conta Transwitch S.Deering Cisoc Systems  N.Gupta, Ed. Tropos Networks
	2006
	Internet Control Message Protocol (ICMPv6) for the Internet Protocol Version 6 (IPv6) Specification
	https://tools.ietf.org/html/rfc4443
\end{thebibliography}